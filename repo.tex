\hypertarget{ux3b5ux3c0ux3b9ux3baux3bfux3b9ux3bdux3c9ux3bdux3afux3b1-ux3b1ux3bdux3b8ux3c1ux3ceux3c0ux3bfux3c5-ux3c5ux3c0ux3bfux3bbux3bfux3b3ux3b9ux3c3ux3c4ux3ae}{%
\section{Επικοινωνία
Ανθρώπου-Υπολογιστή}\label{ux3b5ux3c0ux3b9ux3baux3bfux3b9ux3bdux3c9ux3bdux3afux3b1-ux3b1ux3bdux3b8ux3c1ux3ceux3c0ux3bfux3c5-ux3c5ux3c0ux3bfux3bbux3bfux3b3ux3b9ux3c3ux3c4ux3ae}}

\hypertarget{ux3bbux3acux3b6ux3b1ux3c1ux3b7ux3c2-ux3b1ux3c0ux3bfux3c3ux3c4ux3ccux3bbux3b7ux3c2}{%
\subsection{Λάζαρης
Αποστόλης}\label{ux3bbux3acux3b6ux3b1ux3c1ux3b7ux3c2-ux3b1ux3c0ux3bfux3c3ux3c4ux3ccux3bbux3b7ux3c2}}

\hypertarget{ux3b1ux3bc-ux3c02016059}{%
\subsection{ΑΜ : Π2016059}\label{ux3b1ux3bc-ux3c02016059}}

\hypertarget{my-profile}{%
\subsection{\texorpdfstring{\href{https://github.com/ApoLaz}{My
Profile}}{My Profile}}\label{my-profile}}

\hypertarget{organization}{%
\subsection{\texorpdfstring{\href{https://github.com/Unixidized/Unixidized}{Organization}}{Organization}}\label{organization}}

\begin{longtable}[]{@{}
  >{\raggedright\arraybackslash}p{(\columnwidth - 6\tabcolsep) * \real{0.2500}}
  >{\raggedright\arraybackslash}p{(\columnwidth - 6\tabcolsep) * \real{0.2500}}
  >{\raggedright\arraybackslash}p{(\columnwidth - 6\tabcolsep) * \real{0.2500}}
  >{\raggedright\arraybackslash}p{(\columnwidth - 6\tabcolsep) * \real{0.2500}}@{}}
\toprule()
\begin{minipage}[b]{\linewidth}\raggedright
Εβδομάδα
\end{minipage} & \begin{minipage}[b]{\linewidth}\raggedright
\href{https://courses-ionio.github.io/help/deliverables/}{Όλα τα
παραδοτέα βρίσκονται στην ίδια σελίδα της τελικής αναφοράς} με τα
προσωπικά στοιχεία σας (Όνομα, ΑΜ, github profile) και μαζί με αυτόν εδώ
τον πίνακα περιεχομένων
\end{minipage} & \begin{minipage}[b]{\linewidth}\raggedright
Σύνδεσμος στην
\href{https://github.com/courses-ionio/help/discussions/categories/show-and-tell}{εβδομαδιαία
παρουσίαση προόδου στις συζητήσεις}
\end{minipage} & \begin{minipage}[b]{\linewidth}\raggedright
Αυτοαξιολόγηση σύμφωνα με τα κριτήρια της αντίστοιχης άσκησης
\end{minipage} \\
\midrule()
\endhead
1 &
\href{https://github.com/courses-ionio/hci/discussions/1794}{Δημιουργία
ομάδας} + \href{https://courses-ionio.github.io/help/guide/}{Φορκ και
δημιουργία σελίδας τελικής αναφοράς},
\href{https://raw.githubusercontent.com/courses-ionio/hci/master/README.md}{προσθήκη
πίνακα περιεχομένων},
\href{https://courses-ionio.github.io/help/intro/}{συγγραφή της
εισαγωγής}, αποστολή της εισαγωγής
\href{https://github.com/courses-ionio/help/discussions/categories/show-and-tell}{για
σχολιασμό στην συζήτηση} και καταγραφή του συνδέσμου συζήτησης δίπλα
--\textgreater{} &
\href{https://github.com/courses-ionio/help/discussions/825}{Discussions}
& \\
2 & Άσκηση γραμμής εντολών (arch linux install on VM) &
\href{https://github.com/courses-ionio/help/discussions/1240}{Discussions}
& \\
3 & Συμμετοχικό περιεχόμενο A1 &
\href{https://github.com/courses-ionio/help/discussions/1241}{Discussions}
& \\
4 & Άσκηση γραμμής εντολών (arch linux install on HW) &
\href{https://github.com/courses-ionio/help/discussions/1240}{Discussions}
& \\
5 & Συμμετοχικό περιεχόμενο A2 &
\href{https://github.com/courses-ionio/help/discussions/1242}{Discussions}
& \\
6 & Κατασκευή του βιβλίου Α & & \\
7 & Συμμετοχικό περιεχόμενο B1 & & \\
8 & Άσκηση γραμμής εντολών & & \\
9 & Συμμετοχικό περιεχόμενο B2 & & \\
10 & Άσκηση γραμμής εντολών & & \\
11 & Κατασκευή του βιβλίου Β & & \\
12 & Τελική αναφορά* & & \\
\bottomrule()
\end{longtable}

\hypertarget{ux3c0ux3b1ux3c1ux3b1ux3b4ux3bfux3c4ux3adux3b1}{%
\section{Παραδοτέα}\label{ux3c0ux3b1ux3c1ux3b1ux3b4ux3bfux3c4ux3adux3b1}}

\hypertarget{ux3b5ux3b9ux3c3ux3b1ux3b3ux3c9ux3b3ux3ae}{%
\subsubsection{Εισαγωγή}\label{ux3b5ux3b9ux3c3ux3b1ux3b3ux3c9ux3b3ux3ae}}

\begin{itemize}
\tightlist
\item
  Μέσα από τις ασκήσεις της γραμμής εντολών θεωρώ ότι θα αναπτύξω
  περισσότερο την δυνατότητα να αλληλεπιδρώ με τον υπολογιστή μέσω
  terminal καθώς και να εξοικειωθώ με το dual boot και να ξεφύγω από την
  χρήση των VM. Και θεωρώ ότι μέσα από το κομμάτι των video quiz θα μάθω
  για την ιστορική εξέλιξη των υπολογιστών και της επιστήμης τους κάτι
  το οποίο δεν έχουμε ιδέα το πώς ξεκίνησαν και εξελίχθηκαν μέχρι την
  εποχή όπου η γενιές μας ξεκίνησαν να κάνουν χρήση της τεχνολογίας
  αυτής. Και γνωρίζουμε μόνο την εξέλιξη από εκείνο το επίπεδο μέχρι το
  σημερινό. Και προφανώς επειδή το στιλ το μαθήματος είναι μέσα από το
  github και την αλληλεπίδραση των φοιτητών μεταξύ τους θα αποκτήσω και
  περαιτέρω γνώση πάνω στο github και την δεξιότητα της επικοινωνίας και
  λύσιμο προβλημάτων μέσο αυτού με άλλους προγραμματιστές
\item
  Τελικά έχω καταφέρει μέχρι στιγμής να έχω μια πολύ καλή χρήση του
  συστήματος μου μέσο της γραμμής εντολών αλλά όχι μόνο από στις
  ασκήσεις της γραμμής εντολών όπως είχα πει στην αρχή του εξαμήνου αλλά
  από όλα τα παραδοτέα γιατί όλα τους έχουν να κάνουν με αυτήν και να
  κάνεις της παραμετροποιήσεις που χρειαζόταν για το εκάστοτε παραδοτέο.
  Όσο για τα video quiz θεωρώ ότι έχω ``πέσει'' μέσα γιατί έχω δει πολλά
  πράγματα από το πως ήταν η ενασχόληση των ανθρώπων της επιστήμης της
  πληροφορίας και πως άλλαξε με την εισχώρηση των υπολογιστών στις
  δουλείες του. Όπου δεν χρειαζόταν να μετακινούνται σε όλο το κτίριο
  για να πάρουν κάποια έγγραφα αλλά ούτε να κάνουν χειρωνακτικά τα
  αντίγραφα τους. Είδα τις ιδέες που είχαν να προσθέσουν πάνω στο
  keyboard κάποια macrokeys για να κάνουν την επεξεργασία κειμένου αλλά
  και την χρήση των αρχείων πιο γρήγορη αλλά και συσκευές όπως το
  πληκτρολόγιο ακόρντων που δεν ήξερα καν ότι υπήρχε. Επίσης είδα πως
  και με τι σκεπτικό αναπτύχθηκε το UNIX. Εντόπισα και κατάλαβα τα
  πλεονεκτήματα που μου προσφέρει το τερματικό που δεν τα είχα σκεφτεί
  και εφόσον έχω αυτές τις ``παραπάνω'' γνώσεις που απαιτεί από τον
  χρήστη πλέον το χρησιμοποιώ και στην καθημερινότητα μου. Και με αυτό
  το μάθημα λόγο της φύσης του, με τον οργανισμό που έχουμε αλλά και το
  κομμάτι των discussions έχω εξοικειωθεί με το σκεπτικό να μπω σε ένα
  forum τύπου stackoverflow και να θέσω τα προβλήματα που μπορεί να
  αντιμετωπίζω σε ένα project μου, ακόμα με τα παραδοτέα του
  συμμετοχικού περιεχομένου αφομοίωσα ακόμα καλύτερα τα submodules του
  github από ΄ότι την πρώτη φορά που τα είδα στο μάθημα της τεχνολογίας
  λογισμικού. Και έχω δει και καινούργια πράγματα όπως την lua μέσα από
  την κατασκευή του βιβλίου.
\end{itemize}

\hypertarget{ux3c0ux3b1ux3c1ux3b1ux3b4ux3bfux3c4ux3adux3bf-2}{%
\subsubsection{Παραδοτέο
2}\label{ux3c0ux3b1ux3c1ux3b1ux3b4ux3bfux3c4ux3adux3bf-2}}

\begin{itemize}
\tightlist
\item
  Στο δεύτερο παραδοτέο ``προσπέρασα'' την εγκατάσταση σε Virtual
  Machine και την έκανα κατευθείαν σε USB Stick. Όμως λόγο αυτού πιστεύω
  ότι αντιμετώπισα κάποιες δυσκολίες αλλά με την τρίτη προσπάθεια
  κατάφερα και έκανα μια επιτυχημένη εγκατάσταση. Για την εγκατάσταση
  ακολούθησα τις οδηγίες του Arch Wiki και άλλων πηγών στο διαδίκτυο σε
  διάφορα επίπεδα από το διαχωρισμό του δίσκου μέχρι τα DE και WM. Εγώ
  για αρχή εγκατέστησα ένα DE για να υπάρξει μια πιο γρήγορη
  εξοικείωσης.
\item
  \href{https://asciinema.org/a/OSsmgEqcpg0v3x6VSEarRdpzr}{Neofetch}
\item
  \href{https://asciinema.org/a/jFjUeiKxYpEeuyllmpvzREZwd}{journalctl
  -b}
\end{itemize}

\hypertarget{ux3c0ux3b1ux3c1ux3b1ux3b4ux3bfux3c4ux3adux3bf-3-ux3b11}{%
\subsubsection{Παραδοτέο 3 /
Α1}\label{ux3c0ux3b1ux3c1ux3b1ux3b4ux3bfux3c4ux3adux3bf-3-ux3b11}}

\begin{itemize}
\tightlist
\item
  Για το 3ο παραδοτέο πρόσθεσα στο submodule images 2 εικόνες για το Nes
  Zapper μία κανονικών διαστάσεων και μία μικρή με πλάτος 160px και
  άλλες 2 για το Spyder-IDE αντίστιχα. Και μέσα στο submodule \_gallery
  δημιούργησα 2 .md αρχεία για την προσθήκη λεζάντας στο καθένα
  αντίστοιχα. Έπειτα έκανα τον κατάλληλο συσχετισμό του βασικού repo του
  site με τα submodules για να κάνω το test deploy στο netlify πριν κάνω
  το PR στον οργανισμό. Ακολουθούν τα links.
\item
  \href{https://guileless-mandazi-a0b198.netlify.app/gallery/spyder-ide/}{Spyder-IDE}
\item
  \href{https://guileless-mandazi-a0b198.netlify.app/gallery/nes-zapper/}{Nes
  Zapper}
\item
  \href{https://asciinema.org/a/4Z04Mzts6wUXhoxJhooI5d7JX}{asciinema}
\end{itemize}

\hypertarget{ux3c0ux3b1ux3c1ux3b1ux3b4ux3bfux3c4ux3adux3bf-4}{%
\subsubsection{Παραδοτέο
4}\label{ux3c0ux3b1ux3c1ux3b1ux3b4ux3bfux3c4ux3adux3bf-4}}

\begin{itemize}
\tightlist
\item
  Εγκατέστησα τα Arch Linux όχι σε dual boot αλλά σε ένα USB όπως έχω
  αναφέρει και πιο πάνω στο παραδοτέο 2. Για την κατασκευή του Live usb
  χρησιμοποίησα 2 USB ένα το οποίο περιίχε το Arch Linux iso όπου με το
  Rufus όπου μας προτείνει και το
  \href{https://wiki.archlinux.org/title/USB_flash_installation_medium\#Using_Rufus}{wiki}
  το έκανα να είναι bootable και ένα δεύτερο usb άδειο στο οποίο έγινε η
  εγκατάσταση των Arch. Στην αρχή έκανα boot το installer-usb και με το
  \href{https://wiki.archlinux.org/title/iwd}{iwd} έκανα σύνδεση στο
  δίκτυο για να μπορώ πιο μετά να κατεβάσω τα πακέτα μου καθώς το laptop
  μου δεν διαθέτει θύρα ethernet. Ύστερα προχώρησα με το διαχωρισμό του
  δίσκου(του άδειου usb) όπου το χόρισα σε 2 διαφορετικά partitions το
  ένα για το efi για το \texttt{boot} και το άλλο που θα είναι το root
  partition και τα έκανα format με τις εντολές
  \texttt{mkfs.fat\ -F32\ /dev/sdXn} και
  \texttt{mkfs.ext4\ -O\ "\^{}has\_journal"\ /dev/sdXn1} αντίστοιχα.
  Όπως αναφέροντε και στα
  \href{https://wiki.archlinux.org/title/USB_flash_installation_medium\#In_GNU/Linux_4}{page1}
  και
  \href{https://wiki.archlinux.org/title/Install_Arch_Linux_on_a_removable_medium\#Minimizing_disk_access}{page2}
  του wiki, στα commands όπου Xn και Xn1 είναι τα συγκεκριμένα
  prartiostions όπως τα διαβάζει ο υπολογιστή αυτά μπορείς να τα δείς με
  το command \texttt{lsblk}. Και μετά το format τους τα έκανα mount στα
  /mnt/boot/efi και /mnt. Και κατέβασα στο /mnt την νεότερη έκδοση του
  kernal για τα linux, το firmware για τα linux , το vim για να έχω έναν
  text editor. Ύστερα στο chroot έκανα set το timezone, τα locale που
  θέλω μέσα από το /etc/locale.gen όπου απλά έβγαλα από σχόλιο αυτό που
  με ενδιαφέρει και το έκανα generate με \texttt{locale-gen} σέταρα και
  τη γλώσσα μεσα στο /etc/locale.conf όπου ήταν η επιλογή που έκανα
  πριν, έθεσα το keymap όπου θέλω στο /etc/vconsole.conf, τα hostname
  και hosts, κατέβασα με το pacman κάποια πακέτα όπως το efibootmgr, το
  grub, το networkmanager, το git, το reflector, xdg-utils και
  xdg-users-dirs. Και όπως λέει και στο
  \href{https://wiki.archlinux.org/title/Install_Arch_Linux_on_a_removable_medium\#Installation_tweaks}{wiki}
  μέσα στο /etc/mkinitcpio.conf μετακίνησα τα hooks block και keyboard
  μπροστά από το hook autodetect. Και το έκανα generate με την
  \href{https://wiki.archlinux.org/title/Mkinitcpio\#Image_creation_and_activation}{εντολή}
  \texttt{mkinitcpio\ -p\ linux}. Με την εντολή \texttt{-\/-\/-\/-}
  έκανα install το grub boot-loader. Ενεργοποίησα το Network Manager με
  \texttt{systemctl\ enable\ NetworkManager}. Έφτιαξα τον user μου και
  στη συνέχεια του έδωσα τα δικαιώματα για το sudo. Bγήκα από το chroot
  και έκανα reboot για να κάνω boot το Live USB. Κάνοντας πλέον boot τα
  Arch ακολούθησα τις
  \href{https://wiki.archlinux.org/title/Install_Arch_Linux_on_a_removable_medium\#Minimizing_disk_access}{οδηγίες}.
  Στην συνέχεια με το pacman κατέβασα video drivers. Και κάποια ακόμα
  πακέτα όπως το konsole για το τερματικό μου, το firefox για broswer,
  το KDE για DE. Για την αρχική μου εξοικείωση και είχα σκοπό να
  εγκαταστήσω μεταγενέστερα ένα WM αλλά δεν τα κατάφερα λόγω έλλειψης
  χρόνου.
\item
  Επίσης κατασκεύασα και 2 scripts για 2 από τα εργαλεία που
  προτείνονται για warm-up. Συγκεκριμένα για το youtube-dl για το
  κατέβασμα μουσικής και το wttr για την πρόγνωση του καιρού.
\end{itemize}

\begin{longtable}[]{@{}
  >{\centering\arraybackslash}p{(\columnwidth - 2\tabcolsep) * \real{0.4444}}
  >{\centering\arraybackslash}p{(\columnwidth - 2\tabcolsep) * \real{0.5556}}@{}}
\toprule()
\begin{minipage}[b]{\linewidth}\centering
Χρήσιμα Links
\end{minipage} & \begin{minipage}[b]{\linewidth}\centering
Asciinema
\end{minipage} \\
\midrule()
\endhead
\href{https://archlinux.org/download/}{ISO} &
\href{https://asciinema.org/a/OSsmgEqcpg0v3x6VSEarRdpzr}{Neofetch} \\
\href{https://wiki.archlinux.org/title/Installation_guide}{Installation
page 1} &
\href{https://asciinema.org/a/jFjUeiKxYpEeuyllmpvzREZwd}{journalctl
-b} \\
\href{https://wiki.archlinux.org/title/USB_flash_installation_medium}{Installation
page 2} &
\href{https://asciinema.org/a/0PylBjXEH4m5ohB5XP9kKTUz9}{wttr} \\
\href{https://wiki.archlinux.org/title/Install_Arch_Linux_on_a_removable_medium}{Installation
page 3} &
\href{https://asciinema.org/a/BqQz79DVNWZkRhTbqIhZEwShG}{youtube-dl} \\
\href{https://wiki.archlinux.org/title/iwd}{iwd} & \\
\href{https://wiki.archlinux.org/title/Mkinitcpio}{mkinitcpio} & \\
\href{https://wiki.archlinux.org/title/GRUB}{grub} & \\
\bottomrule()
\end{longtable}

\hypertarget{ux3c0ux3b1ux3c1ux3b1ux3b4ux3bfux3c4ux3adux3bf-5-ux3b12}{%
\subsubsection{Παραδοτέο 5 /
Α2}\label{ux3c0ux3b1ux3c1ux3b1ux3b4ux3bfux3c4ux3adux3bf-5-ux3b12}}

\begin{itemize}
\tightlist
\item
  Για το 5ο παραδοτέο στο repo του site πρόσθεσα στο \_slides στα
  proggraming.mb και videogames.md τα Spyder-IDE και Nes Zapper
  αντίστοιχα. Και στο \_timeline στα programming.md και input-devices.md
  τα Spyder-IDE και Nes Zapper αντίστοιχα. Και όλα αυτά πρώτα τα έκανα
  σε ένα demo branch για να κάνω ένα test deploy στο netlify για να
  σιγουρευτώ ότι δουλεύει πριν κάνω το PR στον οργανισμό. Παρακάτω έχω
  και τα αντίστοιχα links.
\item
  \href{https://guileless-mandazi-a0b198.netlify.app//timeline/programming/}{Spyder-IDE
  timeline}
\item
  \href{https://guileless-mandazi-a0b198.netlify.app//slides/programming/}{Spyder-IDE
  slide}
\item
  \href{https://guileless-mandazi-a0b198.netlify.app//timeline/input-devices/}{Nes
  Zapper timeline}
\item
  \href{https://guileless-mandazi-a0b198.netlify.app//slides/videogames/}{Nes
  Zapper slide}
\item
  \href{https://asciinema.org/a/wOk4ayYCl3tGLo6lGG4apxYve}{Asciinema}
\end{itemize}

\hypertarget{ux3c0ux3b1ux3c1ux3b1ux3b4ux3bfux3c4ux3adux3bf-6}{%
\subsubsection{Παραδοτέο
6}\label{ux3c0ux3b1ux3c1ux3b1ux3b4ux3bfux3c4ux3adux3bf-6}}

\begin{itemize}
\tightlist
\item
  Για το 6ο παραδοτέο αποφάσισα να φτιάξω ένα φίλτρο lua για την
  προσθίκη σχολίων στο βιβλίο. Μετέτρεψά επίσης το script που υπήρχε για
  να αναγνωρίζει το καινούργειο φίλτρο που έφτιαξα, αλλά και να μου
  παράγει το συνολικό latex αρχείο για το βιβλίο χρησιμοποιόντας όλα τα
  latex αρχεία που παράγει το script για όλα τα ξεχωριστά κομμάτια του
  βιβλίου. Και από αυτό το latex έκανα generate το PDF του βιβλίου.
  Προφανώς όλα αυτά γίνανε με την pandoc. Επίσης για την καλύτερη
  οργάνοση του repository μου μάζεψα όλα τα φίλτρα σε έναν φάκελο.
\end{itemize}

\begin{longtable}[]{@{}
  >{\centering\arraybackslash}p{(\columnwidth - 2\tabcolsep) * \real{0.5000}}
  >{\centering\arraybackslash}p{(\columnwidth - 2\tabcolsep) * \real{0.5000}}@{}}
\toprule()
\begin{minipage}[b]{\linewidth}\centering
aciinema
\end{minipage} & \begin{minipage}[b]{\linewidth}\centering
repo info
\end{minipage} \\
\midrule()
\endhead
\href{https://asciinema.org/a/BBSpoeT9j5AirsEwqcjuvMkQb}{script} &
\href{https://github.com/ApoLaz/kallipos/blob/master/make-latex.sh}{make-latex.sh} \\
\href{https://asciinema.org/a/89oXLJNun9DRFQRGcXBsaE9Qz}{lua filter} &
\href{https://github.com/ApoLaz/kallipos/blob/master/screenshot/comment-screenshot.png}{screenshot
of comment} \\
\href{https://asciinema.org/a/wlp9TaLdWDSX4VKPHwZ8bEND8}{creating
comment} &
\href{https://github.com/ApoLaz/kallipos/blob/master/book/book.pdf}{pdf} \\
\bottomrule()
\end{longtable}

\hypertarget{ux3c0ux3b1ux3c1ux3b1ux3b4ux3bfux3c4ux3adux3bf-7-ux3c3ux3c5ux3bcux3bcux3b5ux3c4ux3bfux3c7ux3b9ux3baux3cc-ux3c0ux3b5ux3c1ux3b9ux3b5ux3c7ux3ccux3bcux3b5ux3bdux3bf-b1}{%
\subsubsection{Παραδοτέο 7 (Συμμετοχικό περιεχόμενο
B1)}\label{ux3c0ux3b1ux3c1ux3b1ux3b4ux3bfux3c4ux3adux3bf-7-ux3c3ux3c5ux3bcux3bcux3b5ux3c4ux3bfux3c7ux3b9ux3baux3cc-ux3c0ux3b5ux3c1ux3b9ux3b5ux3c7ux3ccux3bcux3b5ux3bdux3bf-b1}}

\begin{itemize}
\tightlist
\item
  Για τo Β1 συμμετοχικό περιεχόμενο όπου αφορά την προσθήκη μια Μελέτης
  Περίπτωσης αποφάσισα να αναπτύξω περαιτέρω το \href{}{NES Zapper} το
  οποίο το είχα προσθέσει στο 3ο παραδοτέο (Συμμετοχικό περιεχόμενο Α1)
  γιατί θεώρησα ότι μια απλή λεζάντα δεν μπορεί να το καλύψει πλήρως.
  Και πιστεύω ότι αξίζει να αφιερωθεί χρόνο και μελέτη για αυτό, για την
  κατανόηση της τεχνολογίας που είχε και τον πως γινόταν αυτό το
  ``μαγικό'' πράγμα όπου μόνο όταν πετύχαινες τον στόχο αυξανόταν το
  σκορ σου όπου είναι το λογικό αλλά το πως γινόταν αυτό και καταλάβαινε
  ότι τον πέτυχες. Αλλά και να γίνει και μία ιστορική αναδρομή από την
  πρώτη του κυκλοφορία στην Ιαπωνία αλλά και το πως αυτό άλλαξε, σε
  εμφάνιση, μετά την κυκλοφορία του στην Αμερική όπου είναι μια
  διαφορετική κουλτούρα όπου βέβαια δεν έπαιξε μόνο αυτό ρόλο αλλά και
  οι περιορισμοί που τους δημιούργησαν η νομοθεσία του κράτους.
\item
  \href{https://guileless-mandazi-a0b198.netlify.app/case-study/zapper/}{Netlify
  demo}
\item
  \href{https://github.com/Unixidized/site/pull/8}{PR site}
\item
  \href{https://github.com/Unixidized/images/pull/16}{PR images}
\end{itemize}

\hypertarget{ux3c0ux3b1ux3c1ux3b1ux3b4ux3bfux3c4ux3adux3bf-8}{%
\subsubsection{Παραδοτέο
8}\label{ux3c0ux3b1ux3c1ux3b1ux3b4ux3bfux3c4ux3adux3bf-8}}

\hypertarget{ux3c0ux3b1ux3c1ux3b1ux3b4ux3bfux3c4ux3adux3bf-9-b2}{%
\subsubsection{Παραδοτέο 9
(B2)}\label{ux3c0ux3b1ux3c1ux3b1ux3b4ux3bfux3c4ux3adux3bf-9-b2}}

\begin{itemize}
\tightlist
\item
  Για το παραδοτέο αυτό έγραψα την βιογραφία του Seymour Papert. Όπου
  τον εντόπισα σε μια από τις ερωτίσεις των video quiz και είδα ότι δεν
  υπάρχει στο site του βιβλίου και πίστευα ότι είναι μια καλή και
  συμαντική προσθήκη. Καθώς από την ερώτιση του video quiz ακόμα που
  είχα ψάξει για αυτόν είχα δει ότι ήταν μια πολύ ``συμαντική''
  προσωποκότητα. Ο Papert έχει εργαστεί με τον Πιεζέ όπου ήταν και
  μαθητής του και έχει υπάρξει και καθηγιτής στο MIT όπου εκεί έκανε μια
  πολύ συμαντική δουλειά. Ο Seymoyr Papert έχει κατασκευάσει και την
  γλώσσα προγραμματισμού Logo και το Logo Turtle, το οποίο υπάρχει και
  στο βιβλίο μας.
\item
  \href{https://guileless-mandazi-a0b198.netlify.app/biography/saymour-papert/}{Netlify
  demo}
\item
  \href{https://github.com/Unixidized/site/pull/8}{PR site}
\item
  \href{https://github.com/Unixidized/images/pull/16}{PR images}
\end{itemize}

\hypertarget{ux3c0ux3b1ux3c1ux3b1ux3b4ux3bfux3c4ux3adux3bf-10}{%
\subsubsection{Παραδοτέο
10}\label{ux3c0ux3b1ux3c1ux3b1ux3b4ux3bfux3c4ux3adux3bf-10}}

\hypertarget{ux3c0ux3b1ux3c1ux3b1ux3b4ux3bfux3c4ux3adux3bf-11}{%
\subsubsection{Παραδοτέο
11}\label{ux3c0ux3b1ux3c1ux3b1ux3b4ux3bfux3c4ux3adux3bf-11}}

\begin{longtable}[]{@{}l@{}}
\toprule()
Συνεισφορά \\
\midrule()
\endhead
\href{https://github.com/courses-ionio/help/discussions/1156}{Discussion} \\
\bottomrule()
\end{longtable}
